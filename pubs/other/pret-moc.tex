%\documentclass[a4paper,twocolumn]{article}
%\documentclass[10pt, conference, compsocconf]{IEEEtran}
\documentclass[a4paper, conference]{IEEEtran}


\usepackage{pslatex} % -- times instead of computer modern, especially for the plain article class
\usepackage[colorlinks=false,bookmarks=false]{hyperref}
\usepackage{booktabs}
\usepackage{graphicx}
\usepackage{xcolor}
\usepackage{multirow}
\usepackage{cite}
%\usepackage{flushend} % even out the last page, but use only at the end when there is a bibliography

\newcommand{\code}[1]{{\small{\texttt{#1}}}}

% fatter TT font
\renewcommand*\ttdefault{txtt}
% another TT, suggested by Alex
% \usepackage{inconsolata}
% \usepackage[T1]{fontenc} % needed as well?

\usepackage{listings}

\newcommand{\todo}[1]{{\emph{TODO: #1}}}
\newcommand{\martin}[1]{{\color{blue} Martin: #1}}
\newcommand{\abcdef}[1]{{\color{red} Author2: #1}}

% uncomment following for final submission
%\renewcommand{\todo}[1]{}
%\renewcommand{\martin}[1]{}
%\renewcommand{\author2}[1]{}


%%Uncomment the following when you want to add copyright notice and not use any space	 (IEEE only)
%\usepackage[absolute]{textpos}
%% Set unit to be pagewidth and height, and increase inner margin of box
%\setlength{\TPHorizModule}{\paperwidth}\setlength{\TPVertModule}{\paperheight}
%\TPMargin{5pt}
%% Define \copyrightstatement command for easier use
%\newcommand{\copyrightstatement}{
%	\begin{textblock}{0.85}(0.072,0.93)    % Tweak here: {box width}(leftposition, rightposition)
%		\noindent
%		\normalsize
%		???-?-?-???-?/??/\$31.00~\copyright20?? IEEE % Put here your copyright
%	\end{textblock}
%}

\begin{document}

%%Uncomment the following when you want to add copyright notice and not use any space	 (IEEE only)
%\copyrightstatement

\title{A Good Paper Starts with a Good Title}

\author{Marten et.~al}

% Most conferences have their own commands for author headings.

%\author{\IEEEauthorblockN{Edgar Lakis, Martin Schoeberl}\\
%\IEEEauthorblockA{Department of Applied Mathematics and Computer Science\\
%Technical University of Denmark\\
%Email: \texttt{edgar.lakis@gmail.com}, \texttt{masca@imm.dtu.dk}}
%}


\maketitle \thispagestyle{empty}

\begin{abstract}
A good paper has a short, concise abstract. The abstract
states the field of the research, the purpose, and the findings (results).
\end{abstract}

From IEEE: your abstract should:

* Provide a concise summary of the research conducted, the conclusions reached, and the potential implications of those conclusions

* Be self-contained, without abbreviations, footnotes, references, or mathematical equations

* Include 3-5 keywords or phrases that describe the research to help readers find your article

* Consist of a single paragraph of 250 words or less

* Communicate clearly and concisely, with correct grammar and unambiguous terminology

\begin{IEEEkeywords}
real-time systems, time-predictable computer architecture.
\end{IEEEkeywords}



\section{Introduction}
\label{sec:intro}

\todo{A brief introduction what the paper is about. It shall include briefly the
main contributions and findings. The contributions can be bullet listed.}

This paper... \todo{purpose statement, latest in 4th paragraph}

\todo{Test the bib with a reference that gives background on time-predictable
computer architecture~\cite{tpca:jes}.}

A paper is cited \cite{paper:example}.

The contributions of this paper are: (1) ... (2) ...

This paper is organized in N sections: The following section presents related work.
Section~\ref{sec:background} provides background on ...
Section X and Y 
Section~\ref{sec:eval} evaluates...
Section~\ref{sec:conclusion} concludes.

\section{Related Work}
\label{sec:related}

\todo{Show that you know the field. All related work shall be put
into context or contrast to our current work.}

\todo{The following comments from Alessandro Ricci  to Marjan Sirjani have useful pointers to related work.}

- the essential distinguishing point of actors - being them Hewitt's or Edward's -
  is asynchronous message passing. This is not arguable, if we want to talk
  about actors, we need to agree on this.

- the value of asynchronous message passing is nowadays well recognised
   also in the mainstream of distributed systems engineering
   where "uncoupling" is a must
   --- e.g. "reactive manifesto movement", Akka's actors, reactive design patterns, etc.
   --- this is a must in service-oriented architectures community, micro-services, etc.

- the fallacies of adopting RPC / RMI as foundational model has been largely
   recognised in DS literature, even by those who introduced them
   (e.g. "A Note on Distributed Computing", by J. Waldo)

- the problem is coupling model is not only about deadlocks: it has a
   strong impact on reactivity (in a wide sense: https://www.reactivemanifesto.org/)
   on performance, on flexibility

- Said this, of course one could find it useful to adopt RPC as a communication pattern
  on top of the async model, in terms of Request-Response pattern.

- This is largely adopted in the mainstream, e.g.:
    "Reactive Messaging Patterns with the Actor Model: Applications and Integration in Scala and Akka"
    "Reactive Design Patterns" - Roland Kuhn Dr. e Jamie Allen

- It is true that In literature and in mainstream framework, the frequent use of this pattern leads
  to embed it in languages and frameworks, however fully preserving the async msg passing
  and actor semantics.   Examples range from academics (e.g. ActorFoundry) to industrial (Akka "ask").

- Even if the async model is preserved, personally  (given my software engineer background)
  I think that - at a first glance - it is not a good idea, because it  compromises the reasoning
  and methodology in designing actors, mixing async and sync.

- I can do it with agent programming languages, because in that  case it is very clear also
  at the design/methodological level what does it mean mixing apparently sync code within async one.

- Nevertheless, as soon as one defines a clear model/method for conceptually integrating
  sync-like fragments in async code, then this could bring strong benefits.

  In the mainstream for instance, in asynchronous programming frameworks, the "async/await"
  idiom (based on co-routine under the hood) is getting more and more used, instead of
  promises/futures... because it gives a more sync (easy) taste to the code, without creating
  async spaghetti and nested callbacks ("callback hell").

So summarising:

- async msg passing as kernel is not an option today, it is a must
- given that, RPC is naturally recovered as request-response pattern
- if one aims at giving a more sync taste to the code, this may be done
  without compromising the async semantics.

\section{System Model}
\label{sec:sysmod}

\todo{Some RTS conferences like a system model description where we talk
on a set of periodic tasks...}

\section{Background (Maybe)}

\todo{Just with paper links to our work:}

PRET \cite{pret:dac2007} \cite{pret:cases:2008}

Accessors

Patmos/T-CREST \cite{patmos:rts2018} \cite{t-crest:2015}

Rebeca


\section{The PRET Actor Model}
\label{sec:actor}

Our modeling and execution environment consists of actors that contain
handlers. Handlers are executed atomically within an actor.
Handlers can be released be the arrival of messages on the input
port, an external event, an external callback, or by passage of time.

Actors contain input and output ports, which each containing just a single
buffer. Messages can be overwritten and are not consumed on a read.
Each message has a time stamp.

Within actors the logical time does not advance between an input message
and the production of an output message.

\section{Execution Platforms}

For the execution of the actors with repeatable timing we need time-predictable
execution platforms. The basis is a processor with time-predictable or repeatable
timing.

\section{Evaluation}
\label{sec:eval}

\todo{Computer engineering is a constructive science. We build stuff and we measure.
Therefore, there shall always be an evaluation section.}

\subsection{Source Access}

\martin{I love doing papers with available source under an
open-source license. It gives credit and good karma.}

\section{Conclusion}
\label{sec:conclusion}

\todo{Rephrase what this paper is about and list the main contributions and results.}

\subsection*{Acknowledgment}

\todo{Sometimes we received some help. Sometimes external funding.}

%This work was partially funded under the
%European Union's 7th Framework Programme
%under grant agreement no. 288008:
%Time-predictable Multi-Core Architecture for Embedded
%Systems (\mbox{T-CREST}).



\bibliographystyle{plain}
% Please do not add any references to msbib.bib.
% They get lost when I 'generate' is again (see Makefile)
\bibliography{pret-moc,msbib}

\section{Notes}

Here collect of notes and ideas in bullet list for the development and writing process:

\begin{itemize}
\item Collect related work
\item Some more notes
\end{itemize}

\subsection{Terminology}

\martin{We should start to find a terminology that we stick to (within this paper).}

Handlers can be released or fired. Released is more the RTS world, fired more the
actor world.

The data of ports can be called message or event (with data).

\end{document}
